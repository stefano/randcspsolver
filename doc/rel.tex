\documentclass[a4paper,12pt,italian]{article}
\usepackage[italian]{babel}
\usepackage{longtable}
\usepackage{graphicx}
%\usepackage{lastpage}
\usepackage{fancyhdr}

\usepackage{setspace}
%\onehalfspacing

\pagestyle{fancy}
\fancyhead{} %annulla head di default
\fancyfoot{} %annulla foot di default

\newcommand{\titolo}{Ottimizzazione di CSP random con Branch and Bound}
\newcommand{\autore}{Dissegna Stefano, Geremia Mirco}

%testa e piede
\lhead{\leftmark}
%\rhead{}
\cfoot{\thepage}%pagina \thepage\ di \pageref{LastPage}}

\title{\titolo}
\author{\autore}

\begin{document}

\begin{titlepage}
\begin{center}
%\vspace*{1in}
\includegraphics[scale=0.3]{unipd-logo.png} 

\par
\vspace{0.5in}
{\Huge Universit\`a degli studi di Padova}
\par
\vspace{0.5in}
{\LARGE Dipartimento di Matematica Pura ed Applicata}
\par
\vspace{0.3in}
\par
\vspace{0.5in}
{\huge \titolo}
\par
\end{center}
\vspace{0.5in}
\begin{center}
{\Large Studenti: \autore}
\par
\vspace{0.3in}
\end{center}
\par
\vspace{0.5in}
\begin{center}
Anno Accademico 2009-2010
\end{center}

\end{titlepage}

\tableofcontents
\newpage

\section{Scopo del progetto}

Scopo del progetto \`e la risoluzione di problemi di ottimizzazione
espressi sotto forma di CSP binari con variabili a domini finiti. Le
performance del risolutore sono state testate su problemi generati
casualmente. Sono stati messi a confronto i tempi di risoluzione ed il
numero di nodi visitati durante la ricerca di una soluzione ottima tra
un risolutore che non utilizza nessun tipo di propagazione ed uno che
invece mantiene il problema consistente localmente. Il criterio
utilizzato per la consistenza locale \`e la consistenza sugli archi
(AC).

\section{Utilizzo da riga di comando}

Per eseguire il programma si utilizza il seguente comando (il file
\textit{Solver.class} deve trovarsi in una directory presente nella
variabile d'ambiente \textit{CLASSPATH}):
\begin{verbatim}
  java Solver [opzioni]
\end{verbatim}

Il risolutore prevede il passaggio dei parametri da riga di comando secondo
le seguenti regole:

\begin{itemize}
 \item -n N: (N \'e un intero) imposta il numero di variabili rappresentate dal problema;
 \item -l L: (L \'e un intero) imposta la cardinalit\`a dei domini delle varibili;
 \item -d D: (D \'e un reale tra 0 e 1) imposta la densit\`a dei vincoli;
 \item -s S: (S \'e un reale tra 0 e 1) imposta la strettezza dei vincoli;
 \item -ac: se presente, utilizza l'algoritmo AC-1 durante la ricerca.
 \item -b B: (B \'e un intero) se presente esegue il risolutore in
   modalit\`a benchmark. Risolve B problemi generati casualmente e
   stampa in output, in formato CSV, il tempo ed il numero di nodi
   visitati massimi, minimi e medi. In questa modalit\`a l'opzione -m
   viene ignorata. Se non presente, viene generato un solo problema, e
   viene stampato in output il problema ed una sua soluzione ottima.
 \item -m \textit{nomefile.minion}: scrive nel file specificato uno
   script \textit{Minion} che risolve il problema generato casualmente.
\end{itemize}

\subsection{Esempi}

Il seguente esempio genera 100 problemi, ognuno con 10 variabili,
domini di cardinalit\`a 12, con una probabilit\`a 0.7 che un vincolo
tra due variabili sia presente ed una probabilit\`a 0.6 che una coppia
di valori sia accettata da un vincolo. Durante la ricerca della
soluzione ottima utilizza AC come criterio di consistenza locale:

\begin{verbatim}
  java Solver -n 10 -l 12 -d 0.7 -s 0.6 -ac -b 100
\end{verbatim}

Il seguente esempio \`e uguale al precedente, ma genera un solo
problema, lo stampa insieme alla soluzione ottima e genera uno script
equivalente per il risolutore \textit{Minion}:

\begin{verbatim}
  java Solver -n 10 -l 12 -d 0.7 -s 0.6 -ac -m problem.minion
\end{verbatim}

Terminata l'esecuzione \'e possibile risolvere lo stesso problema con
\textit{Minion}:

\begin{verbatim}
  minion problem.minion
\end{verbatim}

\subsection{Compilazione}

Per compilare il programma da sorgenti \'e sufficiente eseguire il
seguente comando nella directory principale del progetto:

\begin{verbatim}
  javac Solver.java
\end{verbatim}

\section{Rappresentazione dei domini}

I domini sono rappresentati dalla classe \textit{ListDomain} in modo
esplicito tramite una lista di valori. Possono quindi essere
rappresentati solo domini finiti. I valori appartenenti ad un dominio
possono essere solamente numeri interi. Non si tratta di una grossa
limitazione in quanto \`e sempre possibile associare univocamente ad
ogni intero un valore. Sui domini \`e possibile compiere
principalmente quattro tipi di operazioni:

\begin{enumerate}
\item Ottenere l'elenco dei valori appartenenti al dominio.
\item Ottenere una copia del dominio, utilizzato dall'algoritmo di
  ricerca per effettuare il backtracking.
\item Rimuovere, in modo distruttivo, dal dominio i valori che non
  sono consistenti con un dato vincolo. La relazione di consistenza
  utilizzata \`e l'AC (arc consistency) direzionale.
\item Controllare se il domino \`e fallito oppure no: un dominio
  fallito \`e un dominio che non contiene valori.
\end{enumerate}

\section{Rappresentazione delle variabili}

Le variabili sono rappresentate dalla classe \textit{Variable}. Ad
ogni variabile \`e associato un nome ed un dominio di valori che pu\`o
assumere . Una variabile a cui \`e stato assegnato un valore viene
rappresentata come una variabile il cui dominio \`e un singoletto
contenente il valore. Le due operazioni principali su una variabile
sono:
\begin{enumerate}
\item Ottenere il suo dominio.
\item Impostare un nuovo dominio.
\end{enumerate}

\section{Rappresentazione dei vincoli}

I vincoli binari sono rappresentati dalla classe
\textit{BinaryConstraint} in modo esplicito. Ogni vincolo \`e composto
da due variabili e da un insieme di coppie di valori accettati tra
queste due variabili. L'insieme \`e implementato tramite una tabella
di hash per poter eseguire velocemente il test di appartenenza. I
vincoli supportano le seguenti operazioni:

\begin{enumerate}
\item Ottenere le due variabili che partecipano nel vincolo.
\item Trasposizione: restituisce un nuovo vincolo uguale all'originale
  ma con le variabili trasposte.
\item Appartenenza: controlla se, data una coppia di valori per le due variabili, la
  coppia appartiene oppure no al vincolo. Data la rappresentazione
  esplicita dei vincoli, il controllo avviene semplicemente
  controllando che la coppia sia presente nell'insieme delle coppie
  del vincolo.
\item Revise: rimuove dalla prima variabile i valori che non sono
  consistenti con la seconda variabile rispetto al vincolo.
\end{enumerate}

\section{Funzione obiettivo ed euristica}

Le funzioni obiettivo ed euristica possono essere passate come
parametri all'algoritmo di ricerca come sottoclassi dell'interfaccia
\textit{Problem.Evaluator} implementando il metodo \textit{eval} che,
data una lista di variabili, ritorna la valutazione. Data la
rappresentazione uniforme delle variabili assegnate e delle variabili
non assegnate, l'interfaccia \textit{Problem.Evaluator} va bene come
base sia per la funzione obiettivo (che operer\`a su variabili i cui
domini sono singoletti) sia per l'euristica (che operer\`a invece su
domini con uno o pi\`u elementi).

\subsection{Funzione massimizzata}

La funzione obiettivo massimizzata durante i test \`e la somma dei
valori delle variabili. L'euristica massimizza quindi la somma dei
valori pi\`u alti dei domini. Se tutte le variabili sono assegnate,
l'euristica coincide con la funzione obiettivo, quindi \`e stata
implementata solo una funzione.

\section{Rappresentazione di un problema}

Un problema \`e rappresentato come:

\begin{itemize}
\item Una lista di variabili.
\item Una lista di vincoli tra le variabili.
\item Una funzione obiettivo.
\item Una funzione euristica.
\end{itemize}

Inoltre un problema mantiene internamente un contatore dei nodi
visitati dall'algoritmo di ricerca.

\subsection{Generazione di un problema}

Il generatore di problemi prende quattro parametri (oltre alle
funzioni obiettivo ed euristica) che utilizza per generare un problema
random:

\begin{enumerate}
\item Numero di variabili del problema. I nomi delle variabili sono un
  numero progressivo.
\item Cardinalit\`a del dominio. Gli elementi di ogni dominio saranno
  numeri interi tra 0 e la cardinalit\`a (esclusa).
\item Densit\`a dei vincoli, un numero compreso tra 0 e 1. Man mano che vengono generati i vincoli
  binari tra tutte le possibili coppie di variabili, un vincolo viene accettato
  con probabilit\`a pari alla densit\`a. Viene generato un numero
  casuale tra 0 e 1, e se non supera la densit\`a allora il vincolo
  viene accettato.
\item Strettezza dei vincoli, un numero compreso tra 0 e 1. Per ogni
  vincolo vengono generate tutte le coppie possibili di valori, ed
  ogni coppia viene accettata con una probabilit\`a pari alla
  strettezza.
\end{enumerate}

La generazione procede nel seguente modo:
\begin{enumerate}
\item Crea un dominio con \textit{D} elementi.
\item Crea \textit{N} variabili. Ad ogni variabile assegna come
  dominio una copia del dominio creato al passo 1.
\item Per ogni coppia di variabili, genera un numero casuale tra 0 e
  1. Se non supera la soglia della densit\`a, crea un vincolo binario
  tra le due variabili. Se \`e stato generato il vincolo, per ogni
  possibile coppia di valori delle due variabili, genera un numero
  casuale tra 0 e 1. Se il numero non supera la strettezza, aggiungi
  la coppia tra le coppie accettate dal vincolo.
\end{enumerate}

\section{Algoritmo di propagazione}

Durante la ricerca viene mantenuta la consistenza sugli archi
(AC). L'algoritmo utilizzato per la propagazione dei vincoli \`e
\textit{AC-1}. L'implementazione opera modificando distruttivamente i
domini delle variabili, quindi durante la ricerca \`e necessario
mantenere delle copie per poterli ripristinare durante il
backtracking. \textit{AC-1} utilizza anche i vincoli trasposti, che
vengono generati prima dell'inizio della ricerca e memorizzati, in
modo da non doverli ricalcolare ogni volta che viene richiamato l'\textit{AC-1}.
Per operare utilizza il metodo \textit{revise} dei vincoli, che
richiama poi \textit{removeInconsistent} sui domini. Questi due metodi
utilizzano il valore di ritorno per indicare se un dominio \`e stato
modificato oppure no durante l'operazione. Se nessun dominio \`e stato modificato dopo
avere effettuato il revise su tutti i vincoli (trasposti e non),
l'algoritmo si ferma in quanto il problema \`e diventato \textit{AC}.

\section{Ricerca}

L'algoritmo Branch\&Bound di ricerca \'e implementato in modo ricorsivo
con l'aiuto di un ciclo while interno, che lavora unicamente sul
dominio della variabile corrente rimuovendone il valore scelto ad ogni iterazione.
La ricerca mantiene la migliore soluzione incontrata fino al passo corrente,
memorizzando i valori scelti per ogni variabile nell'ordine in cui quest'ultime sono analizzate;
alla fine della ricerca si controlla se la variabile che dovrebbe contenere la soluzione
\`e rimasta vuota e in tal caso si avverte l'utente che non esistono soluzioni per il problema dato.
Si tiene inoltre memoria del bound attuale per poterlo confrontare con le
soluzioni raggiunte nel corso dell'algoritmo e poter determinare l'utilit\`a o meno
di intraprendere alcuni cammini. Considerando il problema di massimizzare il risultato,
il bound viene inizialmente impostato al minimo valore che pu\`o avere un intero, ovvero -2$^{31}$, in modo
da far risultare la prima soluzione valida migliore del bound iniziale.\\
Ad ogni nodo raggiunto infatti si calcola la funzione euristica o quella obbiettivo nel caso si tratti
di una foglia dell'albero di ricerca; se la funzione euristica d\`a un
risultato positivo\footnote{Per positivo si intende un valore
  superiore al valore della soluzione corrente, poich\'e
si considera il caso in cui si intende massimizzare il risultato.} 
rispetto al bound attuale allora l'algoritmo viene
richiamato al livello successivo, altrimenti si prova il valore successivo
della variabile corrente o eventualmente si effettua backtracking alla
variabile precedente. Nel caso si stia analizzando una foglia dell'albero, e quindi a tutte
le variabili sia stato assegnato un valore, se la funzione obbiettivo fornisce
un valore superiore al bound corrente e se la soluzione considerata rispetta
tutti i vincoli allora questa viene salvata come soluzione, e il suo valore viene
memorizzato come nuovo bound.\\
La funzione prevede ovviamente anche l'utilizzo della funzione \textit{ac1()}, controllando
un'apposita variabile per sapere se si deve eseguire la propagazione oppure no.
Le funzioni d'appoggio per l'algoritmo son le seguenti:

\begin{itemize}
 \item getBound() e setBound(): per ottenere il valore del bound attuale e per impostarne uno nuovo;
 \item doPropagation(): controlla se \'e richiesta o meno l'AC;
 \item validSol(): controlla se la soluzione raggiunta soddisfa i vincoli e rappresenta quindi
una soluzione valida; 
 \item setSol(): imposta la soluzione corrente come quella migliore raggiunta finora.
\end{itemize}

\subsection{Scelta della variabile}

La scelta della variabile avviene linearmente secondo l'ordine
stabilito nell'array che le contiene, quindi in ordine crescente. 
L'indice di posizione nell'array indica il livello dell'albero in cui viene
rappresentata tale
variabile; il livello viene passato come parametro ad ogni chiamata della
funzione principale \textit{bb()}.

\subsection{Scelta del valore}

I valori vengono scelti in ordine decrescente cos\`i come sono memorizzati
nella variabile che rappresenta il dominio. Nell'operare una scelta di valore
si riduce il dominio di una variabile a quell'unico valore, in modo da
permettere
alle funzioni euristiche e obbiettivo di agire costantemente sui domini.
Nel caso della funzione obbiettivo questa si trover\`a quindi ad analizzare
dei domini singleton, ovvero domini contenenti un solo valore.

\subsection{Ripristinare i domini}

Poich\'e la scelta del valore (oltre all'eventuale propagazione dei
vincoli) avviene modificando distruttivamente i domini delle variabili
\'e necessario, durante il backtracking, provvedere al ripristino di
questi per permettere il corretto avanzamento dell'algoritmo.  Il
dominio della variabile corrente viene copiato all'inizio della
ricorsione mentre la copia di tutti i domini viene effettuata
all'entrata nel ciclo while, tramite il metodo \textit{copy()}.  Alla
fine di ogni iterazione viene ripristinato il dominio della variabile
corrente e quelli delle altre variabili con il metodo
\textit{setDomain()}, in modo da permettere alle sucessive
iterazioni/ricorsioni di agire sui domini corretti.

\section{Test}

Per controllare la correttezza del programma, per ogni problema random
generato \`e possibile stampare l'equivalente script per il risolutore
\textit{Minion}. Questo permette di confrontare la soluzione trovata
da \textit{Minion} con quella del nostro programma.

Per la generazione, \`e stato aggiunto il metodo \textit{toMinion}
alle classi \textit{Variable}, \textit{BinaryConstraint},
\textit{Problem} ed all'interfaccia \textit{Problem.Evaluator}. Questo
metodo genera la rappresentazione dell'oggetto in un formato testuale
adatto ad uno script \textit{Minion}. Per eseguirlo \`e sufficiente
usare il comando:

\begin{verbatim}
  $ minion nome_del_file_generato
\end{verbatim}

\section{Benchmark}
I benchmark sono stati effettuati facendo variare i parametri di densit\`a e strettezza
dei vincoli, stampando per ognuno i valori medi di nodi visitati e tempo
impiegato per eseguire l'algoritmo. Non \`e stato possibile confrontare i risultati
ottenuti con e senza AC in un unico grafico poich\`e i valori ottenuti erano troppo
diversi e questo rendeva l'immagine di scarsa utilit\`a; si \`e deciso quindi di costruire 
per ogni misurazione un grafico del rapporto tra i due algoritmi, per verificare il notevole 
incremento di prestazione ottenuto tramite la propagazione.

\subsection{Densit\`a - Nodi visitati}
In questa sezione si varia il parametro relativo alla densit\`a mettendolo
in comparazione con i nodi visitati; il comando dato per ottenere i grafici sottostanti
\`e il seguente:
\begin{verbatim}
  $ java Solver -n 7 -l 10 -d $d -s 0.5 [-ac]
\end{verbatim}
dove il parametro \textit{\$d} assume i valori compresi tra 0.05 e 1 ad intervalli di 0.05,
e \textit{-ac} viene utilizzato nel secondo grafico per aggiungere l'AC.

\subsubsection{Senza AC}
\includegraphics[scale=0.6]{dens.png}
\\
Come si pu\`o facilmente notare per valori bassi di densit\`a, e quindi con pochi vincoli, l'algoritmo trova
subito la soluzione migliore poich\`e sceglie inizialmente i valori pi\`u grandi per ogni dominio, e
il resto dell'albero viene saltato dato che i valori forniti dalle funzioni euristiche non sono
superiori a quelli della soluzione trovata; quando invece il nuemro dei vincoli comincia ad
aumentare l'algoritmo \`e costretto a continuare la ricerca fino a visitare circa 7 milioni di nodi dei
circa 100 milioni totali dell'albero quando la densit\`a \`e massima.

\subsubsection{Con AC}
\includegraphics[scale=0.8]{densAC.png}
\\
Il grafico relativo al Branch\&Bound con l'aiuto dell'AC evidenzia un picco di quasi 140 nodi visitati
in corrispondenza del valore di densit\`a 0.35; \`e logico pensare che un numero grande di vincoli 
porti ad escludere una parte maggiore dell'albero tramite la propagazione, e infatti si arriva ad un minimo di circa 
30 nodi visitati quando la densit\`a \`e impostata a 1.


\subsubsection{Rapporto (senza AC) / AC}
\includegraphics[scale=0.8]{densNodesQuotient.png}
\\
Il rapporto tra numero di nodi visitati con la ricerca normale rispetto a quello con AC ci illustra
come l'ordine di grandezza per i due algoritmi sia considerevolmente diverso: quando la densit\`a dei vincoli
\`e massima infatti il Branch\&Bound senza AC visita pi\`u di 200000 volte i nodi visitati dalla ricerca con AC.

\subsection{Densit\`a - Tempo impiegato}
Anche in questa sezione il parametro variato \`e quello della densit\`a, conforntandolo per\`o
con il tempo impiegato dall'algortimo ad esaurire la ricerca; il comando dato per ottenere i grafici seguenti
\`e dunque lo stesso della precedente sezione, ovvero:
\begin{verbatim}
  $ java Solver -n 7 -l 10 -d $d -s 0.5 [-ac]
\end{verbatim}
dove il parametro \textit{\$d} assume i valori compresi tra 0.05 e 1.

\subsubsection{Senza AC}
\includegraphics[scale=0.8]{densTime.png}
\\
Questo grafico si comporta in maniera analoga a quello precedente confrontato con i nodi visitati,
con un tempo massimo impiegato di 9 secondi.

\subsubsection{Con AC}
\includegraphics[scale=0.8]{densACTime.png}
\\
Qui invece si pu\`o notare la differenza con il grafico riguardante i nodi, in particolare
non si pu\`o parlare di un vero e proprio picco e si pu\`o constatare comunque che il valore
per cui l'algoritmo richiede pi\`u tempo \`e diverso da quello in cui si denotava la punta massima 
precedentemente; questo ci fa notare che un numero minore di nodi non necessariamente implica una 
ricerca pi\`u veloce.

\subsubsection{Rapporto (senza AC) / AC}
\includegraphics[scale=0.8]{densTimeQuotient.png}
\\
Come nel rapporto precedente anche qui emerge l'importante differenza di tempo che impiegano le
due versioni dell'algoritmo, con quella senza propagazione che diventa pi\`u di 2000 volte pi\`u lenta
in corrispondenza di valori di densit\`a alti.

\subsection{Strettezza - Nodi visitati}
In questa sezione si varia il parametro relativo alla strettezza dei vincoli, analizzandolo
in relazione ai i nodi visitati; il comando dato per ottenere i grafici sottostanti
\`e il seguente:
\begin{verbatim}
  $ java Solver -n 7 -l 10 -d 0.5 -s $s [-ac]
\end{verbatim}
dove il parametro \textit{\$s} assume i valori compresi tra 0.05 e 1 ad intervalli di 0.05.

\subsubsection{Senza AC}
\includegraphics[scale=0.8]{strict.png}
\\
Questo grafico pu\`o essere spiegato analizzando gli effetti della strettezza sulla ricerca:
se in un vincolo sono accettate molte coppie di valori \`e probabile che l'algoritmo trovi
presto la soluzione migliore dato che parte con la scelta dei valori pi\`u alti; se invece sono accettate
poche coppie per vincolo l'algoritmo dovr\`a prloungare la ricerca per trovare una soluzione
che includa le coppie di valori pi\`u alte, rendendo l'euristica meno utile al taglio di alcuni rami
dell'albero. Ecco perch\`e con valori pi\`u bassi di strettezza si arriva a visitare anche pi\`u di 10 milioni 
di nodi.

\subsubsection{Con AC}
\includegraphics[scale=0.8]{strictAC.png}
\\
A differenza del Branch\&Bound normale in questa situazione si nota come l'AC sia
agevolata da una strettezza pi\`u forte (ovvero con valori pi\`u bassi) poich\`e consente
una propagazione pi\`u profonda eliminando molti rami dell'albero di ricerca. Nel grafico si
nota inoltre il picco di nodi visitati in corrispondenza di valori medi di strettezza.

\subsubsection{Rapporto (senza AC) / AC}
\includegraphics[scale=0.8]{strictNodesQuotient.png}
\\
L'andamento del rapporto sopra illustrato \`e molto simile a quello del primo grafico, a testimonianza
dello scarso impatto dei valori dell'algoritmo con AC nel rapporto tra i due, considerando l'ordine di grandezza
notevolmente maggiore dei nodi visitati dal Branch\&Bound normale.

\subsection{Strettezza - Tempo impiegato}
Anche in questa sezione si varia il parametro relativo alla strettezza dei vincoli, confrontandolo
ora con il tempo impiegato; il comando dato per ottenere i grafici sottostanti
\`e dunque uguale al precedente:
\begin{verbatim}
  $ java Solver -n 7 -l 10 -d 0.5 -s $s [-ac]
\end{verbatim}
dove il parametro \textit{\$s} assume i valori compresi tra 0.05 e 1 ad intervalli di 0.05.

\subsubsection{Senza AC}
\includegraphics[scale=0.8]{strictTime.png}
\\
Il risultato rispecchia perfettamente quello del corrispettivo grafico dei nodi visitati e
valgono perci\`o le stesse considerazioni.

\subsubsection{Con AC}
\includegraphics[scale=0.8]{strictACTime.png}
\\
Anche in questo caso, come nel rispettivo precedente, si pu\`o notare che valori bassi
di strettezza sono congeniali a questo tipo di algoritmo; questa volta per\`o non si assiste ad
un picco nella zona centrale, bens\`i ad una stabilizzazione dei risultati medi attorno ai 4ms fino
al valore massimo di strettezza.

\subsubsection{Rapporto (senza AC) / AC}
\includegraphics[scale=0.8]{strictTimeQuotient.png}
\\
Essendo i valori complessivi pi\`u bassi in questa situazione rispetto a quella dove si analizzavano
i nodi visitati, si verifica un lieve discostamento tra il grafico del Branch\&Bound senza AC e questo del rapporto; tuttavia
si pu\`o notare anche in questo caso come per valori bassi di strettezza l'algoritmo con AC sia notevolmente pi\`u
veloce, fino a 12000 volte la versione senza propagazione.

\end{document}
