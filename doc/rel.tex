\documentclass[a4paper,12pt,italian]{article}
\usepackage[italian]{babel}
\usepackage{longtable}
\usepackage{graphicx}
%\usepackage{lastpage}
\usepackage{fancyhdr}

\usepackage{setspace}
%\onehalfspacing

\pagestyle{fancy}
\fancyhead{} %annulla head di default
\fancyfoot{} %annulla foot di default

\newcommand{\titolo}{Ottimizzazione di CSP con Branch and Bound}
\newcommand{\autore}{Dissegna Stefano, Geremia Mirco}

%testa e piede
\lhead{\leftmark}
%\rhead{}
\cfoot{\thepage}%pagina \thepage\ di \pageref{LastPage}}

\title{\titolo}
\author{\autore}

\begin{document}

\begin{titlepage}
\begin{center}
%\vspace*{1in}
\includegraphics[scale=0.3]{unipd-logo.png} 

\par
\vspace{0.5in}
{\Huge Universit\`a degli studi di Padova}
\par
\vspace{0.5in}
{\LARGE Dipartimento di Matematica Pura ed Applicata}
\par
\vspace{0.3in}
\par
\vspace{0.5in}
{\huge \titolo}
\par
\end{center}
\vspace{0.5in}
\begin{center}
{\Large Studenti: \autore}
\par
\vspace{0.3in}
\end{center}
\par
\vspace{0.5in}
\begin{center}
Anno Accademico 2009-2010
\end{center}

\end{titlepage}

\tableofcontents
\newpage

\section{Scopo del progetto}

Risoluzione di problemi di ottimizzazione espressi sotto forma di
problema di vincoli binari con variabili con domini finiti. Le
performance del risolutore sono state testate su problemi generati
casualmente. Il risolutore \`e stato valutato con e senza la
propagazione di vincoli utilizzando AC come criterio di consistenza
locale.

\section{Rappresentazione dei domini}

\section{Rappresentazione delle variabili}

\section{Rappresentazione dei vincoli}

\section{Funzione obiettivo ed euristica}

\section{Algoritmo di propagazione}

\section{Ricerca}
\subsection{Ripristinare i domini}
\subsection{Scelta della variabile}
\subsection{Scelta del valore}

\section{Benchmark}

\subsection{Al variare di ...}
%% grafico + commento testuale

\end{document}
